\documentclass[UTF8]{ctexart}

\usepackage{FulcrumEN}

% margin
\usepackage{geometry}
\geometry{
    paper =a4paper,
    top =3cm,
    bottom =3cm,
    left=2cm,
    right =2cm
}
\linespread{1.0}

\begin{document}
    \begin{center}
        {\LARGE\textbf{Fulcrum Package Usage Example}}

        SJTU AI4Math Team
    \end{center}
    \section{Knowledge Entries}

    \subsection{Color Block Representation}
        
        \begin{axm}
            [TheAxiom]
            {An Axiom}
            [Cat]
            This is a very important axiom.

            In principle, there should be very few axioms.
        \end{axm}
    
        \begin{dfn}
            [TheDefinition]
            {A Definition}
            [Cat]
            This is the content of the definition.
        \end{dfn}
        
        \begin{ppt}
            [TheProperty]
            {A property of the \hyperref[dfn:TheDefinition]{above definition}}
            [Cat]
            This is a property of the \hyperref[dfn:TheDefinition]{above definition}.

            Essentially, properties are also theorems. Generally, property entries follow the \hyperref[dfn:TheDefinition]{definition entry} closely, describing properties that can be naturally deduced from the definition, while more important main theorems are represented by the \hyperref[thm:TheTheorem]{theorem entry}.
        \end{ppt}
        
        \begin{xmp}
            [TheExample]
            {An Example}
            [Cat]
            This is an example serving the \hyperref[dfn:TheDefinition]{above definition}.
            
            Examples are usually added to help readers understand the content of definitions or theorems, and theoretically, removing them does not affect the construction of the main theoretical framework.
        \end{xmp}
        
        \begin{lma}
            [TheLemma]
            {A lemma serving the \hyperref[thm:TheTheorem]{theorem below}}
            [Cat]
            This is a lemma prepared for the \hyperref[thm:TheTheorem]{theorem below}.
            
            In formalist practice, we generally do not encourage referencing content that appears after the current entry. But intuitively, this lemma is prepared for the \hyperref[thm:TheTheorem]{theorem below}, so we reference it here.
        \end{lma}
        
        \begin{thm}
            [TheTheorem]
            {A Theorem}
            [Cat]
            This is a very important theorem.
        \end{thm}
            
        \begin{prf}
            This is a proof of the \hyperref[thm:TheTheorem]{above theorem}.
        \end{prf}
        
        \begin{crl}
            [TheCorollary]
            {A corollary of the \hyperref[thm:TheTheorem]{above theorem}}
            [Cat]
            This is a corollary of the \hyperref[thm:TheTheorem]{above theorem}, but it is not independent or important enough to be written as a separate theorem.
        \end{crl}
        
        \begin{cxmp}
            [TheCounterExample]
            {A Counterexample}
            [Cat]
            This is a counterexample serving the \hyperref[thm:TheTheorem]{above theorem}.
            
            Counterexamples are usually added to help readers understand why certain constraints must be added to a definition or theorem, and theoretically, removing them does not affect the construction of the main theoretical framework.
        \end{cxmp}
        
        \begin{rmk}
            This is a remark, used to explain additional information about a certain entry in natural language.

            We encourage writing semi-formal mathematical language within the main entries, while using remarks to provide intuitive explanations in natural language. 
        \end{rmk}
        
        \begin{pbm}
            Write exercises here. 
        \end{pbm}
        
        \begin{slt}
            Write solutions to the exercises here.
        \end{slt}
        
        \begin{pbm}[3]
            You can reset the exercise counter by providing an optional argument to the \texttt{pbm} environment.
        \end{pbm}

    \section{Lean Language Support}

        \subsection{Code Blocks}
        
        \begin{dfn}
            [Cat]
            {Definition of Cat}
            [Cat]
            Fulcrum is a cat!
            \begin{lstlisting}[style=lean]
    def #tm{Fulcrum} : cat := some random cat
            \end{lstlisting}

            Use \texttt{\#tm\{...\}} to color the declared definition/theorem name.
        \end{dfn}
        
        \begin{thm}
            [CatCuteTheorem]
            {Cat Cute Theorem}
            [Cat]
            All cats are very cute!
            \begin{lstlisting}[style=lean]
    theorem th_name : ∀ (x : cat), x is very cute! := by sorry
            \end{lstlisting}
        \end{thm}

\end{document}